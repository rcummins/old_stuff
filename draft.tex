\documentclass[preprint, authoryear]{elsarticle}
\usepackage{amsmath} % required for basic math equations
\usepackage{wasysym} % required for permil symbol
\usepackage{natbib}

\begin{document}

\begin{frontmatter}

\title{Carbonate clumped isotope constraints on Silurian ocean temperature and seawater $\delta^{18}$O}

\author[caltech]{Renata C. Cummins\corref{cor1}}
\ead{rcummins@post.harvard.edu}

\author[berkley]{Seth Finnegan}

\author[washU]{David A. Fike}

\author[caltech]{John M. Eiler}

\author[caltech]{Woodward W. Fischer}

\cortext[cor1]{Corresponding author}

\address[caltech]{California Institute of Technology, Geological and Planetary Sciences, MC 100-23, Pasadena, CA 91125}

\address[berkley]{Department of Integrative Biology, University of California, 1005 Valley Life Sciences Bldg \#3140, Berkeley, CA 94720}

\address[washU]{Department of Earth and Planetary Sciences, Washington University, St. Louis, Missouri 63130}


\begin{abstract}

Much of what we know about the history of Earth's climate derives from the chemistry of carbonate minerals in the sedimentary record. The oxygen isotopic compositions ($\delta^{18}$O) of calcitic marine fossils and cements have been widely used as a proxy for past seawater temperatures, but application of this proxy to deep geologic time is complicated by diagenetic alteration and uncertainties in the $\delta^{18}$O of seawater in the past. Carbonate clumped isotope thermometry provides an independent estimate of the temperature of the water from which a calcite phase precipitated, and allows direct calculation of the $\delta^{18}$O of the water. The clumped isotope composition of calcites is also highly sensitive to recrystallization and can help diagnose different modes of diagenetic alteration, enabling evaluation of preservation states and identification of the most pristine materials from within a sample set---critical information for assessing the quality of paleoproxy data generated from carbonates. 

We measured the clumped isotope composition of a large suite of calcitic fossils (primarily brachiopods and corals), sedimentary grains, and cements from Silurian (ca. 433 Ma) stratigraphic sections on the island of Gotland, Sweden. Substantial variability in clumped isotope temperatures suggests differential preservation with alteration largely tied to rock-buffered diagenesis, complicating the generation of a stratigraphically resolved climate history through these sections. Despite the generally high preservation quality of samples from these sections, micro-scale observations of calcite fabric and trace metal composition using electron backscatter diffraction and electron microprobe analysis suggest that only a subset of relatively pristine samples retain primary clumped isotope signatures. These samples indicate that Silurian tropical oceans were likely warm (31$\pm$2$^{\circ}$C) and similar in oxygen isotopic composition to that estimated for a ``modern'' ice-free world ($\delta^{18}$O$_{\text{VSMOW}}$ of -1.4$\pm$0.5$\permil$). This result joins the growing body of evidence that suggests the $\delta^{18}$O of Earth's ocean waters has remained broadly constant through time. 
 
\end{abstract}

\end{frontmatter}

\section{Introduction} 

Despite more than half a century of study, the long-term temperature history of Earth's oceans remains poorly constrained. The most widely used proxy for past ocean temperature is the carbonate-water oxygen isotope exchange thermometer, also known as the carbonate $\delta^{18}$O thermometer \citep{Urey1947}. This thermometer is based on the temperature dependence of the fractionation between the $\delta^{18}$O of carbonate, here reported relative to the VPDB standard, and the $\delta^{18}$O of seawater, here reported relative to the VSMOW standard. The carbonate $\delta^{18}$O proxy has two significant complications. First, as with many paleoenvironmental proxies, it is often difficult to distinguish the effects of diagenetic alteration from a primary environmental signature. This is an acute challenge associated with carbonate minerals, which commonly undergo recrystallization during diagenesis and burial, altering primary geochemical signals and impacting the quality of proxy data. Second, the measured $\delta^{18}$O of carbonates depends not only on the temperature of the water from which the carbonates precipitated, but also on the $\delta^{18}$O of the fluid (seawater and/or pore fluids). The $\delta^{18}$O of seawater can be influenced by several processes, including local evaporation and mixing with meteoric water, as well as global changes in the volume and isotopic composition of continental glaciers, all of which are difficult to constrain independently in the geologic record. 

Based largely on observations of carbonates, it was hypothesized that the $\delta^{18}$O of the ocean increased gradually but substantially through Earth's history---by $\sim8\permil$ over Phanerozoic time \citep{Jaffres2007, Veizer1999}. On 100-million-year timescales, gradual changes in $\delta^{18}$O are, in principle, possible given current knowledge of the fluxes impacting the seawater $\delta^{18}$O budget. The long-term average $\delta^{18}$O of all the water on Earth's surface is largely controlled by the balance between two classes of processes within the silicate Earth: exchange during high-temperature alteration of the seafloor near mid-ocean ridges, and exchange during low-temperature weathering of oceanic and continental crust \citep{Muehlenbachs1998}. High-temperature exchange processes have small oxygen isotope fractionation factors and tend to increase the $\delta^{18}$O of ocean water by equilibrating the water with the average mantle $\delta^{18}$O$_{\text{VSMOW}}$ of 5.8$\permil$ \citep{Muehlenbachs1998}. Low-temperature exchange processes have larger oxygen isotope fractionation factors and tend to decrease the $\delta^{18}$O of seawater by preferentially moving $^{18}$O from water into rocks. Furthermore, we might reasonably anticipate that the relative magnitudes of high- and low-temperature weathering have changed over time with continental assembly and rifting \citep{Collins2003}, changes in seafloor spreading and production of oceanic crust \citep{Muller2008, Fornari1995}, climate \citep{Frakes2005}, and sedimentation \citep{Molnar2004}, unless a set of feedbacks, albeit enigmatic, operate to buffer the $\delta^{18}$O of seawater \citep{Gregory1981, Gregory1991, Muehlenbachs1976, Muehlenbachs1998, Jaffres2007}.  

The most widely cited evidence for gradual change in seawater $\delta^{18}$O is an observed ca. 8$\permil$ increase in the average $\delta^{18}$O of marine carbonates from the Cambrian to today \citep{Jaffres2007, Veizer1999}. However, the complications associated with the carbonate $\delta^{18}$O proxy allow conflicting interpretations of this trend in carbonate $\delta^{18}$O, giving rise to a longstanding controversy. Several studies have proposed that early Paleozoic carbonates may have low $\delta^{18}$O values because they are more likely to have undergone diagenetic alteration \citep{Clayton1959, Keith1964, Land1995}, which tends to decrease the $\delta^{18}$O of carbonate \citep{Marshall1992}. In response to this concern, \cite{Veizer1999} measured the $\delta^{18}$O of calcite fossils that were screened for diagenetic alteration using trace metal compositions and micro-scale studies of calcite fabrics. These screened fossils also record increasing $\delta^{18}$O through the Phanerozoic, arguing against diagenetic alteration as a cause of the low $\delta^{18}$O values in the early Paleozoic. An alternative hypothesis holds that oceans were much warmer in early Paleozoic time \citep{Karhu1986, Knauth1976}. Assuming that the $\delta^{18}$O of the oceans was the same as today, Ordovician tropical oceans must have been $\sim$45$^{\circ}$C to produce carbonates with the observed average $\delta^{18}$O$_{\text{VPDB}}$ of -5.9$\permil$. Such warm temperature estimates present an apparent paradox with evidence for a large glaciation in the Late Ordovician \citep{Veizer1986}, leading several authors to suggest that the increase in the $\delta^{18}$O of marine carbonates over time must therefore reflect a concomitant increase in the $\delta^{18}$O$_{\text{VSMOW}}$ of seawater from -8$\permil$ in the Cambrian to its present value of $\sim$0$\permil$ \citep{Jaffres2007, Veizer1999}. 

Carbonate clumped isotope thermometry supplies a unique insight into this problem because it provides data that can help resolve the two main complications of the carbonate $\delta^{18}$O proxy: the clumped isotope composition of a carbonate is independent of the $\delta^{18}$O of water and highly sensitive to diagenetic alteration at temperatures different from those of original deposition. The extent to which $^{13}$C and $^{18}$O occur together, or ``clump'' within the same carbonate ion is a function of the temperature at which the carbonate precipitates \citep{Eiler2011}, and isotopic clumping in a wide diversity of shallow-water calcifiers has been shown to record ambient water temperatures \citep{Ghosh2006, Henkes2013, Saenger2012, Tripati2010}. There is some uncertainty regarding the uniformity of these calibrations, but all previous studies suggest this phenomenon is a reliable and reasonably precise thermometer when calibrated against carbonates of the same type as analyzed in the same lab. Clumped isotope temperature estimates can in turn be used, with the $\delta^{18}$O of shell calcite and the experimentally determined temperature dependence of the oxygen isotope fractionation between water and calcite \citep{Kim1997}, to calculate the $\delta^{18}$O of the water from which the calcite was precipitated.

Fossils that have been deeply buried in lithified sediments can experience alteration at temperatures greater than those of original deposition. This alteration is often associated with recrystallization, which can change the $\delta^{18}$O and trace-metal content of carbonate mineral phases but is frequently difficult to detect. Clumped isotopes are sensitive to this type of alteration because the recrystallized carbonate will take on a new clumped isotope temperature that can be much higher than the temperature of the pristine fossil calcite. If recrystallized material is sampled along with pristine fossil material during analysis (e.g. cements filling void space within a fossil), the clumped isotope temperature of the sample will be significantly higher than the temperature of the seawater where the organism lived. We demonstrate here that carbonate clumped isotopes can provide a more sensitive indicator of diagenetic recrystallization than the $\delta^{18}$O or the trace metal content of a given carbonate phase, particularly in rocks that undergo nearly rock-buffered diagenesis. 

In carbonate strata that have been deeply buried and subjected to high temperatures for extended periods of time, diagenetic alteration can also take the form of closed-system diffusive reordering of atoms within the mineral lattice \citep{Dennis2013, Huntington2011, Quade2013, VanDeVelde2013}. During this process, isotopes of carbon and oxygen diffuse through the calcite mineral lattice, which decreases the degree of clumping and increases the measured temperature of the calcite. The amount of diffusion that occurs is controlled by both the burial temperature and the length of time for which the sample remains at this temperature. The effects of closed-system diffusive reordering can be difficult to recognize, because the reordering does not affect the bulk isotopic composition of the calcite, and it occurs without dissolution and re-precipitation, which means it is also not expected to change the calcite fabric or trace metal content \citep{Eiler2011}. One way to recognize diffusive reordering is through its effects on the distribution of temperatures in a sample suite; diffusive reordering raises the mean temperature without changing the variance in temperature. In addition, the degree of diffusive reordering can be constrained by independent indicators of any given sedimentary succession’s time-temperature history, such as the conodont color alteration index (CAI) or rock evaluation parameters of organic matter maturity. Recent studies of closed-system diffusive reordering suggest the time-temperature conditions required to increase the clumped isotope temperature of a sample by more than the measurement precision of $\sim$1$^{\circ}$C are more than sufficient to change the color of conodont elements, raising the CAI above the pristine value of one \citep{Dennis2010, Epstein1976, Passey2012, Stolper2014}. Therefore, in sample suites with CAI=1 or pre-oil window catagenesis, elevated clumped isotope temperatures can be reasonably interpreted to be the result of diagenetic alteration due to recrystallization. 

In addition to being highly sensitive to diagenetic alteration, carbonate clumped isotope data, when combined with bulk O and C isotope analyses of the same materials, can be diagnostic of specific diagenetic processes and regimes (recrystallization vs. closed-system diffusive reordering; high vs. low water/rock ratio conditions; meteoric vs. marine pore fluids), and can be used to identify the most pristine end-member of a sample set \citep{Eiler2011}. Of particular interest for this study, as carbonates are buried deeper and the temperature increases, the equilibrium oxygen isotope fractionation between water and carbonate grows smaller, preferentially driving $^{18}$O into the water. During rock-buffered diagenesis, this causes the $\delta^{18}$O of the water to increase, while the $\delta^{18}$O of the altered carbonate is not affected. In contrast, during a water-buffered diagenetic regime, increasing temperature instead causes the $\delta^{18}$O of the altered carbonate to decrease. And, diagenesis in the presence of meteoric water usually occurs when the sample is near the Earth's surface, so the clumped isotope temperature of the altered carbonate is cool, but the $\delta^{18}$O of both the carbonate and the water are diagnostically low. We demonstrate here that this distinction has important implications for the estimation of the sea surface temperature and $\delta^{18}$O of seawater in the past.  

\section{Geological setting and materials} 

Samples were collected from several sections exposed on the island of Gotland, Sweden that span the latest Llandovery through Wenlock epochs ($\sim$434 – 430 Ma) of the Silurian Period (Fig. 1). Ages of the formations and correlations between outcrops are constrained by conodont and graptolite biostratigraphy \citep{Jeppsson2006}. The sections record shelf, carbonate platform and backreef lagoon sediments deposited on the margin of the Baltic basin, which was constrained to be $19\pm5.1^{\circ}$ south of the equator during the Silurian Period \citep{Torsvik1992}. 

The oldest exposed strata on Gotland comprise the Lower Visby Formation, which consists of marls interbedded with fine-grained limestones deposited below storm wave base (Fig. 2). Fossils are abundant throughout this unit \citep{Calner2004a, Samtleben1996}. The contact between the Lower and Upper Visby formations is marked by a condensed interval with abundant fossils of the large solitary rugose coral genus \textit{Phaulactis} \citep{Jeppsson1997, Jeppsson2006, Munnecke2003, Samtleben1996}. This contact occurs just above the Llandovery-Wenlock boundary as defined by conodont biostratigraphy \citep{Aldridge1993, Jeppsson1983, Mabillard1985}. We defined the \textit{Phaulactis} bed as a stratigraphic datum to align the sections, except those in the Slite Formation, which are measured relative to an arbitrary bed in the Slite Formation. The contact between the Lower and Upper Visby formations is also characterized by a facies change that represents a maximum flooding interval \citep{Calner2004b}. Carbonate content increases in the Upper Visby Beds, with a decreasing frequency of marl beds and the appearance of bioclastic limestones and reef mounds. Ripple bedforms and cross-stratification indicate deposition above storm wave base. The Upper Visby Formation is particularly fossiliferous \citep{Calner2004a, Samtleben1996}. 

Above the Upper Visby Beds, the H\"{o}gklint Formation consists of algal and crinoidal limestones with some reef mounds, deposited above storm wave base \citep{Riding1991, Samtleben1996, Watts2000}. The contact between the H\"{o}gklint Formation and the overlying Tofta Formation represents a hypothesized sequence boundary \citep{Calner2004b}. The Tofta Formation is characterized by oncoid-rich bedded limestone, and likely represents a shallow restricted setting, such as a backreef lagoon \citep{Riding1991, Samtleben1996}. The overlying Hangvar Formation consists of interbedded marls and limestones with some reef mounds, indicating deposition in a less restricted, deeper-water setting. The youngest strata sampled for this study were from the Slite Formation, which transitions from limestone-rich to marl-rich interbedded limestones and marls, likely representing deposition in deeper water \citep{Calner2004a}. 

The Silurian strata exposed on Gotland present attractive materials for a clumped isotope study. This section has long been recognized for its generally excellent fossil preservation. The strata on Gotland are very nearly flat lying, with a dip of less than 1$^{\circ}$, display only minimal evidence of tectonic disturbance and have not been deeply buried \citep{Calner2004a, Jeppsson1983}. Conodont fossils from this section yield a conodont color alteration index (CAI) of one \citep{Epstein1976, Jeppsson1983}. The Silurian strata on Gotland also have an abundance of fossils, especially brachiopods, throughout the section. We sampled fossil phases for clumped isotope measurements in order to target calcite that was first precipitated directly from seawater (as opposed to carbonate matrix, which may contain authigenic cements). Many studies of paleoclimate have targeted the secondary interior fibrous layer of the brachiopod shell, which is thought to precipitate in equilibrium with seawater and to be relatively resistant to diagenetic alteration \citep{Azmy1998, Samtleben2001}. The fibrous calcite fabric characteristic of the secondary layer has been well characterized in modern and fossil brachiopods \citep{PerezHuerta2007, Samtleben2001, Schmahl2004}, providing a standard of comparison for the samples in our dataset. The trace metal composition of brachiopods has also been extensively studied as an indicator of diagenetic alteration \citep{Azmy1998, Brand2003, Brand2012, Mii1994, Grossman1996, Shields2003}. The other fossils present in this section, including rugose corals, crinoids, bryozoans, and gastropods, are used less often than brachiopods for paleoenvironmental studies, but have been examined by other clumped isotope studies of paleotemperature \citep{Came2007, Dennis2013, Finnegan2011}. Measurements of several fossil taxa allow comparison of the relative preservation of different calcitic materials and can help to detect taxon-specific ``vital effects'' if they are present. 

A final attractive feature of the Silurian section on Gotland is the previously observed positive excursion in $\delta^{13}$C that initiates at the boundary between the Lower and Upper Visby Beds (Fig. 2). This excursion, termed the Ireviken event, is now recognized globally \citep{Munnecke2003}. In Gotland, a 3$\permil$ excursion in $\delta^{13}$C is coupled to a weakly resolved 0.4$\permil$ positive excursion in $\delta^{18}$O \citep{Munnecke2003}. It is somewhat surprising that the positive excursions coincide with a maximum flooding interval, but it is difficult to distinguish the relative contributions of sediment supply and eustatic sea level rise to this facies change \citep{Calner2004a}. The excursions are also broadly coincident with minor extinctions in several benthic and pelagic marine taxa \citep{Munnecke2003}. 

The coincident isotopic excursions and stratigraphic change suggest that the Ireviken Event captures a climate change phenomenon, the nature of which is still debated. The similarity between the isotopic signatures of the Ireviken event and known glaciations, such as the Late Ordovician glaciation, have led several studies to hypothesize a glacial cause for the Ireviken event \citep{Azmy1998, Kaljo2003, Brand2006, Calner2008}. This hypothesis was supported by the presence of glacial tillites in Brazil that were once interpreted to be early Wenlock in age \citep{Grahn1992}, but these diamictites were later re-interpreted as re-workings of earlier glacial deposits. Early Silurian glaciation in South America is currently thought to have ended prior to the late Llandovery, before the Ireviken Event \citep{DiazMartinez2007a, DiazMartinez2007b}. Many of the characteristics of the Ireviken Event, although not necessarily its global nature, could also be explained by a shift from a humid period to an arid period in the tropics, which could accompany a glaciation or occur independently \citep{Bickert1997, Jeppsson1990, Munnecke2003, Samtleben1996}. According to this model, during the humid period recorded by the Lower Visby Formation, increased precipitation would have lowered the $\delta^{18}$O of surface waters and also stimulated the upwelling of deep water with low $\delta^{13}$C onto the shelves. Increased precipitation would have caused intense weathering, which could explain the dominance of marls over carbonate reefs during this time. During the succeeding arid period beginning in the Upper Visby Formation, evaporation would have increased the $\delta^{18}$O of surface waters, and downwelling of saline water would have brought surface water with high $\delta^{13}$C onto the shelves. Low terrigenous influxes would have allowed the growth of carbonate reefs as observed in the H\"{o}gklint Formation. These hypotheses make different predictions for temperature and the oxygen isotopic composition of seawater. Here we test our ability to resolve possible changes in ocean temperature and/or seawater $\delta^{18}$O through the Ireviken event using stratigraphic samples from Gotland. Resolving a change this small with the uncertainties associated with the carbonate clumped isotope thermometer is challenging because a shift of 0.4$\permil$ in $\delta^{18}$O, if caused by change in temperature of carbonate precipitation at constant $\delta^{18}$O of water, corresponds to a change in temperature of approximately 2$^{\circ}$C. Though the uncertainties for a single carbonate clumped isotope temperature are commonly similar to or higher than this, replicate analyses can reduce these errors to $\pm$1$^{\circ}$C or even $\pm$0.4$^{\circ}$C \citep{Thiagarajan2011}.

\section{Methods}

\subsection{Collection and preparation of samples}

Stratigraphic sections were measured and samples were collected from the island of Gotland, Sweden during the summer of 2011. Each sample was labeled with a geographic location, as shown in Fig. 1, and a stratigraphic height. For locations where the \textit{Phaulactis} layer is exposed, stratigraphic height was expressed relative to the boundary between the Lower and Upper Visby formations (Fig. 2). Other locations were correlated on the basis of prior sequence-, bio- and chemostratigraphic work \citep{Calner2004a}, as well as new bulk carbonate $\delta^{13}$C measurements \citep{Fike2014}.

Of the 22 brachiopods chosen for sampling, 20 belonged to the species \textit{Atrypa reticularis}. We targeted this species because it is abundant in the sampled strata, the shell has a thick secondary fibrous layer, and it was used in several previous paleoenvironmental studies in Gotland \citep{Bickert1997, Munnecke2003, Samtleben2001}. 

Most of the fossils were easily exposed by breaking the surrounding layers of poorly-lithified marl. For fossils in more strongly lithified beds, fossil material was accessed by cutting away the rock with a saw or microrotary drill. After fossils were separated from the rock, they were sonicated for two hours in a Cole Parmer 8891 ultrasonic cleaner to remove residual sand, silt and mud from the surface of the fossils. Brachiopods, bryozoans, crinoids, gastropods and ostracods were sampled by flaking the surface of the skeleton or shell using a dental chisel. Every flake was visually inspected under a stereomicroscope at ca. $25\times$ magnification. Flakes with visible pyrite grains or rough-textured recrystallized regions were rejected and not included in the material for analysis. For brachiopods, only the flakes preserving the characteristic fabric of the fibrous secondary layer were chosen for analysis (see Fig. 3). Flakes were then sonicated briefly, rinsed, dried in a 50$^{\circ}$C oven, and powdered using a mortar and pestle. Only the apical ends of rugose corals were sampled, after removal of any sediment from the surface using an abrasive bit. The apical end was then cut off using a diamond microsaw, and powdered using a mortar and pestle. The apical ends of the corals were targeted because the coral septa are closely spaced together there, leaving less room for sediment and cement between the septa. However, this sampling technique is still not as selective as the technique used for brachiopods, and can include more cement and sediment along with skeletal calcite. 

All fossil samples shown in figures and used to estimate Silurian seawater temperature and $\delta^{18}$O were prepared as described above, with the aim of minimizing the inclusion of cement, sediment and altered material. However, in a few separate test cases we purposefully sampled cements, micrite and other textures that are independently known to be more susceptible to diagenetic alteration, as a way to assess possible diagenetic end-members. Micrite and cement samples were drilled from cut surfaces and then powdered using a mortar and pestle. During the preparation of one brachiopod sample, flakes with large rough-textured recrystallized regions were picked and analyzed separately from the pristine flakes of fibrous secondary layer as a measure of alteration. 

While all fossils were inspected under a microscope prior to sampling, we were not able to inspect all fossils in thin section, because carbonate clumped isotope measurements require significant amounts of material, some of which is lost during preparation of thin sections. However, 30 $\mu$m-thick uncovered, polished thin sections were obtained from cuts through four representative fossils, two brachiopods and two rugose corals. These thin sections were inspected first in transmitted and reflected light to characterize calcite textures, and were then examined using several different micro-scale spectroscopic techniques, described below. 

\subsection{Electron microscopy}

Electron backscatter diffraction (EBSD) provides a measure of the degree and style of recrystallization by mapping the orientation of calcite crystals in a thin section \citep[e.g.][]{Bergmann2013}. This allows more detailed characterization of calcite fabrics than is possible using optical microscopy, and in the case of brachiopods, comparison to pristine modern examples \citep{PerezHuerta2007}. Two thin sections, one rugose coral and one brachiopod, were analyzed using a Zeiss 1550VP Field Emission Scanning Electron Microscope (SEM) under variable pressure vacuum, with an accelerating voltage of 20 kV. The stage was tilted 70$^{\circ}$. The AZtec program was used in 4x4 binning mode, with frame averaging over one frame and a step size of 2 $\mu$m for the rugose coral and 1.2 $\mu$m for the brachiopod. The data were analyzed using software from Oxford instruments.

Flakes of the secondary fibrous layer from twelve different brachiopods were also imaged via electron microscopy to define their morphology (but not orientation; i.e., we did not perform EBSD on these flakes). This further inspection helped us assess the preservation of original fibrous calcite texture at a finer scale than available from the reflected light microscope. The rinsed and dried flakes were prepared for imaging by coating them with a layer of carbon 12 nm thick. The coated flakes were imaged using the secondary electron detector in the same SEM as above, under high vacuum. The working distance was $\sim$6 mm and the accelerating voltage was 10 kV.

\subsection{Wavelength-dispersive spectroscopy}

The use of wavelength-dispersive spectroscopy on x-rays emitted during electron microprobe analysis, in combination with optical microscopy and EBSD, allows characterization of the trace metal contents of specific calcite textures. By comparing the trace metal concentrations of primary textures to those of diagenetic textures, we can identify the trace metal signature of diagenetic alteration in our sample set. The same thin sections as above were coated with a layer of carbon 15 nm thick, and then analyzed with wavelength-dispersive spectroscopy using the JEOL JXA-8200 Electron Probe Micro-analyzer (E-probe). Using the JEOL map analysis program, the beam was calibrated on crystal standards of dolomite, siderite, strontianite and rhodochrosite. Qualitative maps of Ca, Mg, Fe, Sr and Mn concentration were made using a current of 200 nA, voltage of 15 kV, and a counting time of 80 milliseconds per pixel. Quantitative spot measurements were made using the JEOL quantitative analysis program with the same beam conditions as used for the qualitative maps and counting times of 20 seconds on the elemental peak and 10 seconds on background. 

\subsection{Bulk trace metal analysis}

The trace metal concentrations of bulk samples can be used as an indicator of the amount of sediment and recrystallized material included in the sample \citep{Azmy1998, Brand2012, Came2007, Finnegan2011}. Most studies of the trace elements in fossil calcite measure concentrations of iron (Fe), manganese (Mn), and strontium (Sr), because diagenetic recrystallization is thought to enrich calcite in Fe and Mn and deplete Sr from calcite \citep{Brand1980, Shields2003}. We measured the bulk concentrations of these three elements in a large and representative subset of our samples. Between 1 and 5 milligrams of each powdered sample were dissolved in a known volume of 4\% nitric acid (HNO$_3$). Standards were prepared from standard solutions of Sr, Mn and Fe with the same concentration of Ca and HNO$_3$ as the sample solutions to minimize matrix effects. Solutions were sampled using a Cetac ASX 260 autosampler, aspirated into the Ar plasma using a peristaltic pump, and then analyzed on a Thermo Scientific iCAP 6000 series inductively-coupled plasma optical emission spectrometer (ICP-OES). Between each set of ten unknown samples, three standard solutions (10 ppb, 100 ppb, and 1 ppm) were measured. Reproducibility was on average 3\% for Fe, 2\% for Mn and 3\% for Sr. 

\subsection{Clumped isotope analysis and standardization}

For each carbonate clumped isotope measurement approximately 10 milligrams of powdered sample were dissolved in phosphoric acid at 90$^{\circ}$C, which produces CO$_2$ with a known oxygen isotopic offset from the sample carbonate. The subsequent purification of evolved CO$_2$ followed procedures outlined by \cite{Passey2010}. Cleaned CO$_2$ was analyzed on a Finnigan MAT253 gas source mass spectrometer configured to collect masses 44 through 49. The sample gas was measured for eight acquisitions of seven cycles each, where each cycle was bracketed by a measurement of reference CO$_2$ with known $\delta^{13}$C and $\delta^{18}$O. The $\delta^{13}$C, $\delta^{18}$O, $\delta_{45}$, $\delta_{46}$, $\delta_{47}$, and $\delta_{48}$ were calculated for each sample relative to the reference CO$_2$. Raw measured values for $\Delta_{47}$ and $\Delta_{48}$ of each sample, defined as $\Delta_{47-[\text{SGvsWG}]}$ and $\Delta_{48-[\text{SGvsWG}]}$, were also calculated relative to the reference CO$_2$ following equations 3 and 4 in \cite{Huntington2009}. Each sample was measured between one and three times depending on the amount of available material.

All $\Delta_{47}$ values are projected into the absolute reference frame \citep{Dennis2011}. The absolute reference frame is based on a thermodynamic equilibrium established in two types of CO$_2$ gases: heated gases and water-equilibrated gases. Heated gases are aliquots of CO$_2$ that were sealed in quartz tubes, heated at 1000$^{\circ}$C for three hours, and then cooled rapidly to room temperature. Water-equilibrated gases are aliquots of CO$_2$ that were equilibrated with water at 25$^{\circ}$C for at least 24 hours. Heated gases were made from two separate reservoirs of CO$_2$ with different $\delta^{18}$O compositions, and water-equilibrated gases were equilibrated with either of two separate waters, also differing in their $\delta^{18}$O compositions. Heated and water-equilibrated gases were cleaned and measured following the same procedure as the samples.

Before sample $\Delta_{47}$ values could be projected into the absolute reference frame, we first needed to establish accepted values in the absolute reference frame for two California Institute of Technology intralaboratory carbonate standards. $\Delta_{47}$ values of the two standards were converted to the absolute reference frame using only data from weeks during which heated gases and water-equilibrated gases were measured along with the standards. For each of these weeks, a transfer function was constructed by plotting the accepted absolute reference frame $\Delta_{47}$ values of 0.0266$\permil$ for heated gases and 0.9252$\permil$ for 25$^{\circ}$C water-equilibrated gases against their observed $\Delta_{47}$ as determined in the \cite{Ghosh2006} reference frame \citep{Dennis2011}. This transfer function was used to convert the $\Delta_{47}$ values of the two intralaboratory carbonate standards from the \cite{Ghosh2006} reference frame to the absolute reference frame following equation C1 in \cite{Dennis2011}.

The $\Delta_{47}$ values of all of the samples in this study were projected into the absolute reference frame using a separate transfer function constructed for each week of measurements. For weeks when water-equilibrated gases were measured, this transfer function was constructed by plotting the accepted absolute reference fame $\Delta_{47}$ values of the two intralaboratory carbonate standards, heated gases, and 25$^{\circ}$C water-equilibrated gases against their observed $\Delta_{47}$ as determined in the \cite{Ghosh2006} reference frame. For weeks when water-equilibrated gases were not measured, this transfer function included only measurements of the two intralaboratory carbonate standards and heated gases. $\Delta_{47}$ values of the samples in the \cite{Ghosh2006} reference frame were corrected using this transfer function following equation C1 in \cite{Dennis2011}. The resulting $\Delta_{47-\text{RF}}$ was corrected for the acid extraction at 90$^{\circ}$C by adding 0.092$\permil$ \citep{Henkes2013} to produce the final value ($\Delta_{47-\text{RF-AC}}$).

Although all samples were cleaned prior to introduction into the mass spectrometer, the possibility for contamination from hydrocarbons and chlorocarbons still exists. To screen for contamination, the $\delta_{48}$ and $\Delta_{48}$ of each sample were compared to a line of best fit through a plot of $\delta_{48}$ versus $\Delta_{48}$ of the heated gases run during the same week, all in the \cite{Ghosh2006} reference frame. Samples whose $\Delta_{48}$ fell more than 1$\permil$ off the line were discarded due to likely contamination. 

$\Delta_{47-\text{RF-AC}}$ values were converted to temperature using the \cite{Ghosh2006} calibration line, which is projected into the absolute reference frame in equation 9 in \cite{Dennis2011}. The $\delta^{18}$O of water was calculated from the temperature and the measured $\delta^{18}$O of the sample using the temperature-dependent fractionation, $\alpha$, of oxygen isotopes between water and calcite, $1000\ln\alpha=18.03(10^3\text{T}^{-1})-32.42$, where T is temperature in degrees Kelvin \citep{Kim1997}. 

One-sigma standard errors based on all acquisitions in each measurement range from 0.0007$\permil$ to 0.0065$\permil$ (averaging 0.0017$\permil$) for $\delta^{13}$C, from 0.0008$\permil$ to 0.0069$\permil$ (averaging 0.003$\permil$) for $\delta^{18}$O, and from 0.005$\permil$ to 0.019$\permil$ (averaging 0.012$\permil$) for $\Delta_{47}$. These errors in $\Delta_{47}$ are consistent with counting-statistics limits of 0.013--0.016$\permil$ \citep{Huntington2009}. The errors in $\Delta_{47}$ propagate into temperature errors ranging from $\pm$1.2$^{\circ}$C to $\pm$4.4$^{\circ}$C, with an average of $\pm$2.8$^{\circ}$C. 

\section{Results and discussion}

\subsection{Stratigraphic trends}

The $\delta^{13}$C values of fossil material, micrite and cement are highly consistent within beds (Fig. 4). We observe a $\sim$3.5$\permil$ positive excursion in $\delta^{13}$C, beginning just below the boundary between the Lower and Upper Visby formations, and continuing for $>$50 meters. Our samples do not capture the relaxation of this excursion, but samples from the Slite Formation all have consistently lower $\delta^{13}$C$_{\text{VPDB}}$ values of approximately -0.5$\permil$. The $\delta^{13}$C trend through the section is similar to the brachiopod $\delta^{13}$C measurements from \cite{Bickert1997} and \cite{Munnecke2003}. 

The $\delta^{18}$O values of fossil material are consistent within beds, but micrite and cements tend to have lower $\delta^{18}$O values than fossil material, often several permil lower than fossils in the same bed (Fig. 4). It is interesting that we were not able to resolve an excursion in carbonate $\delta^{18}$O. This may simply be because our dataset contains fewer samples than that of \cite{Munnecke2003}, although even with the additional samples, the previously reported excursion is poorly resolved. No $\delta^{18}$O excursion was observed in the isotopic data generated from bulk sediment collected from the same sections \citep{Fike2014}. The average $\delta^{18}$O$_{\text{VPDB}}$ value of the fossils remains fairly constant at -5.5$\permil$ through the Lower and Upper Visby formations. These values are extremely similar to those of \cite{Munnecke2003} for the Lower Visby Formation, but our data only overlap with the lower end of the $\delta^{18}$O$_{\text{VPDB}}$ values from \cite{Munnecke2003} for the Upper Visby Formation. The low $\delta^{18}$O values of Upper Visby fossils may be caused by increased contamination from secondary cements, which could be expected to accompany the transition from marl to limestone. Higher in the section, the $\delta^{18}$O values of the fossils in this study are consistent with values reported by \cite{Bickert1997}. 

Clumped isotope temperatures of samples are highly variable both between and within beds. Of the eight beds for which we measured more than one fossil sample, four beds have temperature ranges of more than 13$^{\circ}$C (Fig. 4). Assuming an error of 5$^{\circ}$C to account for environmental variability and measurement error, we used order statistics to calculate that there is a probability of less than 1\% that one sample in these beds will have a temperature more than 13$^{\circ}$C higher than the true depositional temperature, if all samples record a primary temperature. Therefore, the within-bed scatter in clumped isotope temperatures suggests variable preservation quality among these samples. Even the secondary fibrous shell layer of brachiopods, which is commonly thought to be relatively resistant to diagenetic alteration, displays a range in temperature of more than 15$^{\circ}$C across the whole dataset. The variable preservation quality of these samples means that beds with fewer samples are less likely to contain a pristine fossil with a reliable temperature---introducing a sample bias. Because many of the beds in this study are only represented by a few samples, we are not able to stratigraphically resolve what we would interpret robustly as a change in seawater temperature with time. 

\subsection{Preliminary considerations of the diagenetic regime}

Although we are not able to measure a change in sea surface temperature through the Ireviken Event, we can use our data to constrain more broadly temperature and $\delta^{18}$O of seawater by uncovering the mechanism of diagenetic alteration to identify the most pristine samples. As described above, diagenetic regimes can vary in their fluid composition (meteoric vs. marine pore fluids) and in their mechanism (closed-system diffusive reordering vs. recrystallization and incorporation of late-stage cements). We suggest that alteration likely occurred in the presence of marine pore fluids rather than meteoric fluids, because all of the fossil samples have fairly high $\delta^{18}$O values, making low-temperature alteration by meteoric water unlikely. We also argue that closed-system diffusive reordering did not strongly affect any of our samples. Gotland strata have a conodont color alteration index of one \citep{Epstein1976, Jeppsson1983}, which constrains the possible effects of diffusive reordering to an increase in the clumped isotope temperature by a maximum of $\sim$1$^{\circ}$C \citep{Dennis2010, Passey2012, Stolper2014}. Closed-system diffusive reordering is also incompatible with the observed variability in clumped isotope temperatures because this process is expected to preserve the small variance resulting from environmental variability and measurement uncertainty (Fig. 5). The variance of our sample set is much larger than the maximum measurement uncertainty of 4.4$^{\circ}$C, which is consistent with recrystallization as the dominant diagenetic process. This leads us to hypothesize that the lowest temperature samples are the most pristine, while the higher temperature samples likely represent recrystallization and incorporation of late-stage cements during burial at elevated temperatures.

Comparisons of the clumped isotope temperature of different calcite textures within samples help diagnose the diagenetic regime that acted to increase the apparent temperature of more altered samples. We measured the clumped isotope composition of micrites adjacent to four brachiopods, and in all four cases, the temperature of the micrite was higher than the temperature of the neighboring brachiopod (Fig. 6). Micrite is expected to be less pristine than the brachiopod samples because it has high surface areas, far more commonly recrystallizes during burial, and often includes several generations of cements. Because the less pristine material, micrite, was associated with higher temperatures, these comparisons are consistent with the hypothesis that the lowest temperature samples are the most pristine. Interestingly, while the $\delta^{13}$C values of micrites are similar to those of adjacent brachiopods, three of the four micrites tested are depleted in $\delta^{18}$O relative to the adjacent brachiopods, suggesting that the micrites reflect recrystallization with a higher water-to-rock ratio.

In a similar experiment, recrystallized skeletal material was also shown to be associated with higher clumped isotope temperatures. One brachiopod sample was split into two fractions during picking. The first fraction consisted of pristine flakes of fibrous secondary layer and had a temperature of 38$\pm$3$^{\circ}$C, while the other fraction was comprised of flakes with large rough-textured recrystallized regions and had a temperature of 43$\pm$3$^{\circ}$C, where the error estimates represent the one-sigma standard error of the mean over all acquisitions. The results of this experiment support for the hypothesis that the lowest-temperature samples are the most pristine.  

\subsection{Independent assessments of preservation quality}

We tested the hypothesis that the lowest-temperature samples are the most pristine using independent methods, described above, to assess the preservation quality of representative samples. Bulk trace metal measurements, which have been widely used as indicators of diagenetic alteration \citep{Azmy1998, Brand2012, Came2007, Finnegan2011}, suggest that all the brachiopods in this study are comparatively pristine. However, micro-scale studies of calcite fabrics, as well as clumped isotope measurements, demonstrate that there is a range in preservation quality among the brachiopods that cannot be identified on the basis of trace metal measurements alone. We first explore possible reasons for the insensitivity of trace metal concentrations to preservation quality. Then we consider EBSD and SEM evidence for the inclusion of diagenetic material in the typical fossil sample. 

Clumped isotope thermometry implies variable preservation among the brachiopods in this study, so we expected to observe equivalent variations in trace metal content. Diagenetic recrystallization of brachiopod skeletal calcite is thought to enrich Fe and Mn because these metals are insoluble in oxic seawater but often found at elevated concentrations in anoxic pore fluids \citep{Brand1980, Shields2003}. Diagenetic recrystallization is also thought to deplete Sr because brachiopod skeletons concentrate Sr relative to its abundance in seawater and meteoric waters \citep{Brand1980, DePaolo2011, Schrag1995}. Previous studies have defined limits for the Mn and Sr concentrations of brachiopods that can be considered pristine \citep{Azmy1998, vanGeldern2006, Korte2005, Mii1999}. The bulk Mn and Sr concentrations of all brachiopods in this study fall within these proposed limits (Fig. 7); the small variations in trace metal concentrations among the brachiopods would not ordinarily be interpreted as variable preservation. 

We also expected to find a correlation between clumped isotope temperatures and trace metal evidence for diagenetic alteration, as observed for fossils by \cite{Finnegan2011}. We tested brachiopods, rugose corals, and micrites/cements separately and combined for the robustness of a linear correlation between temperature and the log of trace metal concentrations (Fig. 8). We performed a hypothesis test on each pair of variables to test the probability (\textit{p}) that the observed correlation is due to random variation when the true correlation is zero. We used a Bonferroni correction to adjust the significance level ($\alpha$) to account for the multiple hypothesis tests. There are weakly significant positive correlations between temperature and Fe concentration and also between temperature and Mn concentration when all samples are combined. These correlations are largely due to differences between fossil taxa and later phases; brachiopods have the lowest average temperature, Fe, and Mn concentrations, while micrites and cements combined have the highest average temperature, Fe and Mn concentrations. There is also a weakly significant negative correlation between temperature and Sr concentration among rugose corals, which strengthens (to 99\% confidence) when corals are combined with brachiopods, and with micrites and cements. Trace metal concentrations reflect broad differences between taxa and textures, however none of the other variable pairs are significantly correlated within individual taxa (Supplementary table 2). These tests may be complicated by contamination from early void-filling cements, which have low temperatures but ``altered'' trace metal signatures because they precipitate from pore fluids. Variability in Mn concentrations of low-temperature samples may obscure a positive correlation between temperature and Mn concentration, but low-temperature contamination does not appear to explain the lack of correlation between temperature and Fe or Sr. From these observations we suggest that trace metal concentrations, perhaps with the exception of Sr in rugose corals, are relatively insensitive indicators of alteration within fossil taxa under mild burial diagenesis, and in this case are only sensitive enough to reflect broad differences between taxa. 

There are several possible explanations for the relative insensitivity of trace metal concentrations to diagenetic alteration of these materials. Our sample set likely does not include a sufficient number of highly altered samples to capture the full range in severity of alteration, because visibly altered fossils were rejected during visual screening of samples and not included in the sample set. Variability in primary trace metal concentrations, which is large among modern brachiopods \citep{Brand2003}, may also mask any trends superimposed by diagenetic alteration. However, E-probe analysis suggests a more compelling reason for the insensitivity of trace metal concentrations: diagenetic recrystallization need not strongly affect Fe and Mn concentrations. A recrystallized region of a rugose coral septum is clearly visible as a dark area in transmitted light (Fig. 9A) and an area of aligned calcite crystals in an electron backscatter diffraction (EBSD) image (Fig. 9B). However, the Fe and Mn concentrations of the recrystallized region are indistinguishable from those of the pristine coral skeleton (Fig. 9C \& D). This is only one example, but it highlights relatively large calcite domains within samples that were recrystallized but not strongly modified in trace metal concentrations. The trace metal signature associated with recrystallization likely depends on conditions during recrystallization, suggesting that this coral underwent closed-system diagenesis, or recrystallization in the presence of trace-metal-poor waters. If closed-system recrystallization is an important contributor to the elevated temperatures of fossil material in our sample set, trace metals may not be expected to be reliable indicators of diagenetic alteration. 

We therefore turn to micro-scale studies of calcite textures as indicators of diagenetic alteration. Reflected-light and EBSD images of rugose coral fossil skeletons reveal substantial domains of late-phase diagenetic calcite, including recrystallized regions and cements (Fig. 9A \& B). As discussed above, recrystallized regions are visible as dark colors in reflected light and aligned calcite crystals in EBSD images. Cements in Fig. 9A \& B are characterized by large, subequant, interlocking calcite crystals. 

In contrast to the rugose corals, the EBSD image of a low-temperature (35$^{\circ}$C) brachiopod thin section demonstrates excellent preservation of original fibrous calcite textures (Fig. 10A). Within the brachiopod shell, calcite crystals are elongated approximately parallel to the surface of the shell, as expected for the fibrous secondary layer. This preferential orientation is represented by the vertical band in the stereographic projection of crystallographic planes (Fig. 10B). No large regions of recrystallization are visible within the shell. There are large interlocking cement crystals in the upper left hand corner of the EBSD image, but these formed in the void space between the two valves, and would not have been included in the sample for clumped isotope analysis. Based on visual inspection of the EBSD image of this brachiopod's secondary fibrous layer, we suggest that this sample contains a maximum of 5\% recrystallized material. 

Scanning electron microscope (SEM) images of low-temperature brachiopod flakes also demonstrate excellent preservation of original fibrous calcite textures (Fig. 3A, B \& C). The original fibrous texture clearly contrasts with the rough recrystallized texture covering part of a flake that would not have been chosen for analysis (Fig. 3D). The difference between these two textures is visible under the microscope used for picking brachiopod flakes, so flakes with large recrystallized regions were easily identified and excluded from the sample. However, small recrystallized regions, especially if they occurred between layers of fibrous texture, may elude identification during picking. Therefore, although the representative flakes chosen for inspection with the SEM appear pristine, this does not guarantee that all recrystallized material was excluded from the samples.

The micro-scale studies of calcite textures in Figs. 3, 9 and 10 reveal relatively well-preserved samples, but do not rule out the inclusion of small amounts of late-phase calcite. Variable fractions of late-phase calcite likely contribute to the range in temperature among fossil samples. Due to constraints on the amount of available material for analysis, we only performed SEM and EBSD analysis on a small subset of samples, but an important conclusion that can be drawn from the electron microscopy and spectroscopy results is that the low temperature brachiopods imaged in Figs. 3 and 10 probably contain at most only trace amounts of late-phase calcite, variable but likely less than 5\% of the sample. We therefore look to where these samples fall on diagenetic trajectories to estimate Silurian ocean temperature and the value of seawater $\delta^{18}$O. 

\subsection{Constraining the temperature and $\mathit{\delta^{18}}\textit{O}$ of Silurian oceans} 

To estimate the temperature and $\delta^{18}$O of Silurian oceans, we first need identify the diagenetic regime responsible for the observed differential alteration of the samples. As outlined above, different diagenetic processes have characteristic effects on the relationship between temperature and the $\delta^{18}$O of interstitial water. When the clumped isotope temperatures of the samples in this study are plotted versus the $\delta^{18}$O of water calculated for each sample, there is a clear positive relationship (Fig. 11). This relationship suggests that interstitial water grew increasingly enriched in $^{18}$O as the temperature increased, which is characteristic of rock-buffered diagenesis. Importantly, during rock-buffered diagenesis, more altered samples have higher temperatures, which is consistent with the hypothesis that the lowest temperature samples are the most pristine.

The lowest-temperature sample within our sample set, a brachiopod (G13 10.5 to 10.8 B1, Slite Fm.), suggests that tropical Silurian oceans had a temperature of 31$\pm$2$^{\circ}$C and a $\delta^{18}$O$_{\text{VSMOW}}$ of -1.4$\pm$0.5$\permil$ (Fig. 11). The two next-most pristine samples, also brachiopods (G11 0.2 to 0.4 B6, stratigraphic height 1.3 m; G2 0.5 to 0.6 B2, stratigraphic height 0.5 m), have temperatures of 33$\pm$2$^{\circ}$C and 34$\pm$1$^{\circ}$C. Together with the $\delta^{18}$O$_{\text{VPDB}}$ of each brachiopod, these temperatures can be used to calculate seawater $\delta^{18}$O$_{\text{VSMOW}}$ values of -2.2$\pm$0.4$\permil$ and -1.2$\pm$0.2$\permil$, respectively. One caveat of these values is that they are each based on only one measurement due to sample-size limitations. The lowest-temperature brachiopod was picked a second time, but to collect enough material for analysis, some lower quality flakes had to be included, which yielded a temperature of 41$\pm$4$^{\circ}$C (G13 10.5 to 10.8 B1 leftovers). 

Due to the increase of observed sample temperatures during diagenesis, our estimates of the temperature and $\delta^{18}$O of Silurian tropical oceans are upper limits. We are not able to exclude the possibility that small amounts of diagenesis affected all the samples in our study, even the most pristine samples. However, micro-scale studies of the calcite fabric of the lowest-temperature brachiopods indicate that late-phase calcite is only a trace constituent of these samples, comprising a maximum of 5\% of the material. Assuming closed-system diagenesis so that the late-phase calcite is similar in bulk $\delta^{13}$C and $\delta^{18}$O to the pristine fossil calcite, $\Delta_{47}$ values will mix linearly \citep{Affek2006}. In this case, if the measured temperature of the sample is 31$^{\circ}$C and 5\% of the sample material was recrystallized at a maximum of 80$^{\circ}$C (constrained by CAI=1), the primary temperature of the pristine fossil material would be 29$^{\circ}$C. Therefore, the constraints provided by the EBSD mapping of the brachiopod shell calcite demonstrate that our estimate of the temperature of the tropical Silurian ocean is likely reliable to within 2$^{\circ}$C---similar to our measurement uncertainty.

These clumped isotope results indicate that Silurian tropical oceans were broadly similar to modern tropical oceans in their temperature and $\delta^{18}$O. Our estimate of the temperature at a Silurian paleolatitude of $19\pm5.1^{\circ}$ south \citep{Torsvik1992} is within 1$^{\circ}$C of the maximum observed temperature for the modern ocean at a similar latitude \citep{Reynolds1994}. The average $\delta^{18}$O$_{\text{VSMOW}}$ of the modern ocean is approximately -0.3$\permil$ \citep{Shackleton1974}, but without ice, the average $\delta^{18}$O$_{\text{VSMOW}}$ would decrease to approximately -1.4$\permil$ \citep{Lhomme2005}. As stated above, the volume of continental ice is not known with certainty for all of early Silurian time, but assuming Silurian ice volume was less than or equal to modern ice volume, our findings suggest that the $\delta^{18}$O of Silurian oceans was within 2$\permil$ of the modern value. 

The results of this study are also consistent with those of an earlier clumped isotope study. \cite{Came2007} measured the clumped isotope compositions of brachiopods from the early Silurian on Anticosti Island, Canada, and report temperatures of 30 to 35$^{\circ}$C and seawater $\delta^{18}$O$_{\text{VSMOW}}$ values of around -1$\permil$, which are similar to the results of this study. Both of these sedimentary basins were bathed in tropical seawater, but each reflects a different inland sea (Anticosti Island on the craton margin of a foreland basin, and Gotland on the southern margin of the Baltic Shield). The strong similarity in temperatures and $\delta^{18}$O between these localities lends some confidence to paleoclimate reconstruction during Silurian time. 

Our results rule out large ($>2\permil$) long-term changes in the $\delta^{18}$O of seawater and extremely hot ($>40^{\circ}$C) ocean temperatures as causes of the depleted $\delta^{18}$O values of Silurian carbonates. Instead, clumped isotopes demonstrate that even among samples with well-preserved calcite fabrics and pristine trace metal signatures, there is a large range in preservation quality. This means that the average fossil calcite in almost any sample suite is likely altered from its primary composition in an asymmetric way. We note that it is common to take the average $\delta^{18}$O value of carbonate samples when producing time series records of paleoclimate \citep{Bickert1997, Jaffres2007, Munnecke2003, Shields2003, Veizer2000}. If a carbonate sample set has undergone high-temperature fluid-buffered diagenesis, causing more altered samples to have lower $\delta^{18}$O values, the average $\delta^{18}$O of all the samples would not accurately reflect primary ocean conditions. The best-preserved sample in our dataset has a $\delta^{18}$O$_{\text{VPDB}}$ of -5.0$\permil$, which is close to the average $\delta^{18}$O$_{\text{VPDB}}$ value (-4.6$\permil$) of compiled low-magnesium calcites from the Silurian Period \citep{Veizer1999}; however the same need not be true of all sample sets or intervals, particularly as one moves into older and more deeply buried strata. Indeed, the most pristine sample from the upper Mississippi Valley (pre-glacial Ordovician) measured by \cite{Finnegan2011} has a $\delta^{18}$O$_{\text{VPDB}}$ of -4.9$\permil$, while the average of compiled low-magnesium calcites measured globally from the Ordovician Period is -5.9$\permil$ \citep{Veizer1999}. 

The results from this study join a growing set of clumped isotope studies \citep{Came2007, Dennis2013, Finnegan2011} which suggest that the $\delta^{18}$O$_{\text{VSMOW}}$ of seawater has been broadly constant throughout much of Phanerozoic time, at around -1$\pm$1$\permil$ (Fig. 12). Several studies have hypothesized that high- and low-temperature alteration of the seafloor acts as a buffer for the $\delta^{18}$O of seawater \citep{Gregory1981, Gregory1991, Muehlenbachs1976, Muehlenbachs1998}. Observations from modern ocean drill cores suggest that the average temperature of exchange between the seafloor and seawater is 250 to 350$^{\circ}$C \citep{Muehlenbachs1998}. At these temperatures, an ocean with a $\delta^{18}$O$_{\text{VSMOW}}$ of $\sim$0$\permil$ is in isotopic equilibrium with the 5.8$\permil$ mantle. However, this buffering capacity rests on the assumption that the flux of $^{18}$O exchange due to continental weathering has remained insignificant in the overall $^{18}$O budget throughout Earth’s history, and also that the average temperature of exchange between the seafloor and seawater has not changed through time. More recent theoretical efforts challenged these assumptions and suggested that this buffer may not be strong enough to prevent slow changes in the $\delta^{18}$O of seawater \citep{Jaffres2007}. If this buffer cannot explain the relative constancy of seawater $\delta^{18}$O values through time, then the two competing controls on seawater $\delta^{18}$O, alteration of rocks at high and low temperatures, must have remained in balance over the last 500 million years. This suggests that the magnitudes of high-temperature hydrothermal alteration and low-temperature weathering have not changed substantially, or that any changes in the magnitude of one process are tied through as-yet unidentified feedbacks to balance changes in the other. 

\section{Conclusions}

We used clumped isotopes to constrain both the temperature and $\delta^{18}$O of Silurian oceans because this method provides a temperature that is independent of the $\delta^{18}$O of the water from which the carbonate precipitated. Samples were collected from well-preserved flat-lying sections on the island of Gotland, Sweden. By sampling well-preserved, fossil-rich rocks that potentially record a climate phenomenon associated with changing seawater temperatures, we tested the ability of clumped isotopes to resolve potentially small climate change phenomena associated with the Ireviken Event. Despite the overall high quality of fossiliferous materials in our Gotland sections, we were not able to resolve a robust secular change in temperature through the section, due to the variable preservation quality of the samples. We observed that trace metal concentrations have only a limited ability to detect variations in preservation among Gotland brachiopods, despite a range in brachiopod clumped isotope temperatures, suggesting that clumped isotopes are a more sensitive indicator of diagenetic alteration than trace metal compositions. This conclusion is supported by EBSD observations of fossil calcite that was recrystallized without a detectable change in trace metal concentrations. Micro-scale analysis of calcite fabrics, comparisons between brachiopods and micrite, and the positive relationship between temperature and $\delta^{18}$O of water calculated from the samples all support the hypothesis that the lowest-temperature samples are the most pristine with alteration of fossil material largely tied to rock-buffered diagenesis. Based on the most pristine materials in our dataset, we estimate that tropical Silurian oceans had a temperature of 31$\pm$2$^{\circ}$C and a $\delta^{18}$O$_{\text{VSMOW}}$ of -1.4$\pm$0.5$\permil$. Our results, along with those from other clumped isotope studies of paleoenvironment, suggest that the $\delta^{18}$O of seawater (considering the fraction stored as ice during the Pleistocene) has remained broadly constant over Phanerozoic time. 

\section*{Acknowledgements}

We acknowledge Nami Kitchen, Kristin Bergmann, Daniel Stolper, and the rest of the Eiler lab for help with the clumped isotope measurements and J. Garrecht Metzger for assistance in field collection. We thank Chi Ma for his assistance with the SEM and the E-probe, and Joel Hurowitz for help with trace element geochemistry. We also thank Mark Garcia and David Mann for their help with sample processing and Clint Cummins for help with statistics. This work was supported by NSF grant EAR-1053523 and the Agouron Institute.

\section*{References}
\bibliographystyle{my-geochim}
\bibliography{draft}

\end{document}
